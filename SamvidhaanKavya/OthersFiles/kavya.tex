% phpMyAdmin LaTeX Dump
% version 4.6.4
% https://www.phpmyadmin.net/
%
% Host: 127.0.0.1
% Generation Time: Sep 02, 2018 at 11:17 AM
% Server version: 5.7.14
% PHP Version: 5.6.25
% 
% Database: 'sanvidhan'
% 

%
% Structure: kavya
%
 \begin{longtable}{|l|c|c|c|l|} 
 \caption{Structure of table kavya} \label{tab:kavya-structure} \\
 \hline \multicolumn{1}{|c|}{\textbf{Column}} & \multicolumn{1}{|c|}{\textbf{Type}} & \multicolumn{1}{|c|}{\textbf{Null}} & \multicolumn{1}{|c|}{\textbf{Default}} & \multicolumn{1}{|c|}{\textbf{Comments}} \\ \hline \hline
\endfirsthead
 \caption{Structure of table kavya (continued)} \\ 
 \hline \multicolumn{1}{|c|}{\textbf{Column}} & \multicolumn{1}{|c|}{\textbf{Type}} & \multicolumn{1}{|c|}{\textbf{Null}} & \multicolumn{1}{|c|}{\textbf{Default}} & \multicolumn{1}{|c|}{\textbf{Comments}} \\ \hline \hline \endhead \endfoot 
\textbf{\textit{SlNo}} & int(3) & No &  \\ \hline 
Part & varchar(2) & No &  \\ \hline 
kavya1 & varchar(300) & No &  \\ \hline 
kavya2 & varchar(300) & No &  \\ \hline 
anu & varchar(300) & No &  \\ \hline 
 \end{longtable}

%
% Data: kavya
%
 \begin{longtable}{|l|l|l|l|l|} 
 \hline \endhead \hline \endfoot \hline 
 \caption{Content of table kavya} \label{tab:kavya-data} \\\hline \multicolumn{1}{|c|}{\textbf{SlNo}} & \multicolumn{1}{|c|}{\textbf{Part}} & \multicolumn{1}{|c|}{\textbf{kavya1}} & \multicolumn{1}{|c|}{\textbf{kavya2}} & \multicolumn{1}{|c|}{\textbf{anu}} \\ \hline \hline  \endfirsthead 
\caption{Content of table kavya (continued)} \\ \hline \multicolumn{1}{|c|}{\textbf{SlNo}} & \multicolumn{1}{|c|}{\textbf{Part}} & \multicolumn{1}{|c|}{\textbf{kavya1}} & \multicolumn{1}{|c|}{\textbf{kavya2}} & \multicolumn{1}{|c|}{\textbf{anu}} \\ \hline \hline \endhead \endfoot
1 & 1 & भाग-1 & संघ और उसका राज्य क्षेत्र & (The Union and its Territories) \\ \hline 
2 & 1 & सब राज्यों को मिलाकर, भारत एक संघ होगा, & चाहें तो, नवराष्ट्रों का मिलन, विधि सम्मत होगा। & अनुच्छेद - 1,2) \\ \hline 
3 & 1 & संसद चाहे, नये राज्यों का निर्माण कर सकती है, & वर्तमान राज्यों में, परिवर्तन भी कर सकती है। & (अनुच्छेद - 3) \\ \hline 
4 & 2 & भाग-2 & नागरिकता & (Citizenship) \\ \hline 
5 & 2 & भारत में रहने वाला व्यक्ति, भारतीय कहलाता है, & बाहर से आ बसे भारत में, वह भी ये हक पाता है। & (अनुच्छेद - 5) \\ \hline 
6 & 2 & कोई स्वेच्छा से करे, विदेशी नागरिकता स्वीकार, & भारतीय नागरिकता पर, वह खोयेगा अधिकार। & (अनुच्छेद - 9) \\ \hline 
7 & 3 & भाग-3 & मूल अधिकार  & (Fundamental Rights) \\ \hline 
8 & 3 & मानव मूल अधिकारों को, महत्वपूर्ण बतलाया है, & इनके विरुद्ध किसी विधि को, असंगत ठहराया है। & (अनुच्छेद - 13) \\ \hline 
9 & 3 & अनुच्छेद चौदह में, समता का अधिकार है, & होटल हो चाहे दुकान, सबको प्रवेश अधिकार है। & (अनुच्छेद - 14, 15) \\ \hline 
10 & 3 & अनूसूचित जाति, जनजाति की करनी होगी भलाई & ताकि उनकी उन्नित हो और कमी की हो भरपाई। & (अनुच्छेद - 15) \\ \hline 
11 & 3 & नौकरी में धर्म, जाति, लिंग का नहीं होगा भेद, & साफ साफ शब्दों में, समझाता सोलह अनुच्छेद। & (अनुच्छेद - 16) \\ \hline 
12 & 3 & समाज के पिछड़े वर्गों को,हो आरक्षण का सहारा, & उन्नित करें दलित, जिनका जीवन वंचित सारा। & (अनुच्छेद - 16) \\ \hline 
13 & 3 & छुआछूत कलंक का, अनुच्छेद सतरह करे अंत, & डॉ० अम्बेडकर हुए मसीहा, मानवता के बने संत। & (अनुच्छेद - 17) \\ \hline 
14 & 3 & रहे ना कोई जमींदार अब, ना नवाब ना महाराजा, & पद ना होगा जन्म से, खुला योग्यता का दरवाजा। & (अनुच्छेद - 18) \\ \hline 
15 & 3 & प्रतियोगिता से ही मिलेगा, अब पद और सम्मान, & बनो डॉक्टर और कलक्टर, खूब कमाओ नाम। & (अनुच्छेद - 18 [भाव की अभिव्यक्ति]) \\ \hline 
16 & 3 & व्यक्त करो अपने भावों को, जो चाहे सो बोलो, & अन्याय से नहीं डरो, मुँह तुम अपना खोलो। & (अनुच्छेद - 19) \\ \hline 
17 & 3 & संघ और समूह बनाना, है सबका अधिकार, & शांतिपूर्ण करो सम्मेलन, जो हो बिना हथियार। & (अनुच्छेद - 19) \\ \hline 
18 & 3 & सारा भारत देश तुम्हारा, कहीं भी जा सकते हो, & कोई जगह पसन्द आये, वहाँ भी रह सकते हो। & (अनुच्छेद - 19) \\ \hline 
19 & 3 & बनो वकील या व्यापारी, करो कोई भी कारोबार, & अनुच्छेद उन्नीस देता है, तुमको ये सारे अधिकार। & (अनुच्छेद - 19) \\ \hline 
20 & 3 & जब तक सिद्ध नहीं हो दोष,दोषी नहीं कहलाओगे, & बेगुनाह साबित होने के, सब अवसर तुम पाओगे। & (अनुच्छेद - 20) \\ \hline 
21 & 3 & पहले कारण बतलायेंगे, तभी करेंगे गिरफ्तार, & चौबीस घंटे में छोड़ेंगे, गिरफ्तारी यदि बेआधार। & (अनुच्छेद - 22) \\ \hline 
22 & 3 & कोई बच्चा ना करे मजदूरी, मजदूर ना करे बेगार, & यदि कोई भी करे उल्लंघन, जाना होगा कारागार। & (अनुच्छेद - 23) \\ \hline 
23 & 3 & नन्हें - नन्हें से बच्चे, नहीं करे खान में काम, & सरकार उन्हें देगी शिक्षा, ताकि बने बड़े इंसान। & (अनुच्छेद - 24, 21 क) \\ \hline 
24 & 3 & किसी भी धर्म को मानो, या उसका प्रचार करो, & अपने बनाओ संस्थान, संचालन का काम करो। & (अनुच्छेद - 25, 26) \\ \hline 
25 & 3 & कर से होता मुक्त है, धार्मिक कार्यों का प्रबंध, & सरकारी स्कूलों में है, धार्मिक शिक्षा पर प्रतिबंध। & (अनुच्छेद - 27, 28) \\ \hline 
26 & 3 & बौद्ध मुस्लिम सिख ईसाई,अल्पसंख्यक हैं ये भाई, & अपने धर्म, लिपि, भाषा की, कर सकते हैं पढ़ाई। & (अनुच्छेद - 29, 30) \\ \hline 
27 & 3 & मूल अधिकार बहुत जरूरी, प्रजातंत्र का सार, & इन सब की गारंटी देता, संवैधानिक उपचार। & (अनुच्छेद - 32) \\ \hline 
28 & 3 & छीने यदि कोई अधिकार, रिट दायर करवाओ, & न्यायालय से लो आदेश, लागू उसे कराओ। & (अनुच्छेद - 32) \\ \hline 
29 & 3 & शस्त्र बलों व आई०बी० में,बहुत जरूरी अनुशासन, & अधिकारों में अल्प कटौती, कर सकता है शासन। & (अनुच्छेद - 33) \\ \hline 
30 & 4 & भाग-4 & राज्य के नीति निर्देशक तत्व & (Directive Principles of State Policy) \\ \hline 
31 & 4 & संविधान सब राज्यों को, देता है निर्देश, & जनता के कल्याण की, व्यवस्था करें विशेष। & (अनुच्छेद - 38) \\ \hline 
32 & 4 & सामाजिक असमता कम हो, घटे आय का अंतर, & सभी क्षेत्र करें विकास, अवसर मिले निरन्तर। & (अनुच्छेद - 38) \\ \hline 
33 & 4 & स्त्री-पुरुष सबको मिले, आजीविका के साधन, & सामूहिक हित हो जिसमें, बाँटो सब संसाधन। & (अनुच्छेद - 39) \\ \hline 
34 & 4 & एक हाथ में केन्द्रित ना हों, उत्पादन के साधन, & नर-नारी यदि करें कार्य, तो मिले समान ही वेतन। & (अनुच्छेद - 39) \\ \hline 
35 & 4 & कार्मिकों के स्वास्थ्य का, ध्यान रखे सरकार, & स्त्री - पुरुष, बच्चे न करें, हानिकारक रोजगार। & (अनुच्छेद - 39) \\ \hline 
36 & 4 & सब बच्चे उन्नित करें, मिले माहौल अनुकूल, & शोषण उनका ना होये, कभी न जाना भूल। & (अनुच्छेद - 39) \\ \hline 
37 & 4 & न्याय सर्वजन को मिले, अपराधी या सुशील, & कोई बहुत गरीब हो, तो मुफ्त में मिले वकील। & (अनुच्छेद - 39 क) \\ \hline 
38 & 4 & गाँव में निश्चित करें, पंचायत का गठन, & जनता को शक्ति मिले, लगे विकास में धन। & (अनुच्छेद - 40) \\ \hline 
39 & 4 & बूढ़ा हो, बेकार हो, असमर्थ या बीमार, & सहारा इनको भी मिले, कुछ मदद करे सरकार। & (अनुच्छेद - 41) \\ \hline 
40 & 4 & अच्छा, सुविधाजनक हो, कार्यालय में काम, & गर्भवती महिलाओं को, मिले विशेष आराम। & (अनुच्छेद - 42) \\ \hline 
41 & 4 & जीवन का स्तर सुधरे, चहुंमुखी होये विकास, & बेकारों को मिले काम, सुविधा और अवकाश। & (अनुच्छेद - 43) \\ \hline 
42 & 4 & उद्योगों के प्रबन्ध में, हो मजदूरों का हाथ, & सहकारी समिति का स्वरूप, खिले साथ ही साथ। & (अनुच्छेद - 43 क और ख) \\ \hline 
43 & 4 & सब नागरिकों के लिए, जो देश में करें निवास, & समान सिविल संहिता का, किया जाये प्रयास। & (अनुच्छेद - 44) \\ \hline 
44 & 4 & नन्हें बच्चों को मिले, शिक्षा और सुरक्षा, & बनें सभ्य नागरिक वे, जीवन हो अति अच्छा। & (अनुच्छेद - 45) \\ \hline 
45 & 4 & अनुसूचित जाति, जनजाति, वर्ग हैं सबसे दुर्बल, & अन्यायों पर लगे रोक, मिले शिक्षा और धन बल। & (अनुच्छेद - 46) \\ \hline 
46 & 4 & मिलें विटामिन, पोषक तत्व, उच्च स्तरीय जीवन, & स्वास्थ्य सेवा में हो सुधार, रुके नशे का सेवन। & (अनुच्छेद - 47) \\ \hline 
47 & 4 & वैज्ञानिक तकनीकों से बढ़े, कृषि और पशुपालन, & सुधरे नस्ल पशुओं की, वध-निषेध का हो पालन। & (अनुच्छेद - 48) \\ \hline 
48 & 4 & पयार्वरण सुरक्षित रखो, करो जीवन का रक्षण, & वृक्ष लगाओ, करो हरियाली,वन जीवों का रक्षण। & (अनुच्छेद - 48 क) \\ \hline 
49 & 4 & स्मारकों की करें सुरक्षा, रखें विशेष ध्यान, & इतिहास रहेगा सुरक्षित, बढ़ेगा सबका ज्ञान। & (अनुच्छेद - 49) \\ \hline 
50 & 4 & प्रशासन का न्यायालय में, ना हो कोई दखल, & न्यायपालिका रहे स्वतंत्र, न्याय मिले प्रतिपल। & (अनुच्छेद - 50) \\ \hline 
51 & 4 & सब देशों के बीच में, रहे सुरक्षा शान्ति, & आपस में आदर बढ़े, मिटे परस्पर भ्रान्ति। & (अनुच्छेद - 51) \\ \hline 
52 & 4 & भाग-4क & मूल कर्तव्य & (Fundamental Duties) \\ \hline 
53 & 4 & संविधान का करें पालन, झंडे का करें सम्मान, & इसके आदर्शों को माने, आदर से करें राष्ट्रगान। & (अनुच्छेद - 51 क [ क ]) \\ \hline 
54 & 4 & आजादी के आदर्शों को, समझे और बयान करें, & देश की प्रभुता, अखंडता के, रक्षक का काम करें। & (अनुच्छेद - 51 क [ ख, ग ]) \\ \hline 
55 & 4 & राष्ट्र की सेवा करें, स्त्री का सम्मान करें, & भेदभाव रहित होकर, भ्रातृत्व का निर्माण करें। & (अनुच्छेद - 51 क [ घ, ङ ]) \\ \hline 
56 & 4 & सामाजिक संस्कृति समझें, परम्परा का मान करें, & प्रकृति का विस्तार करें, जीवन का सम्मान करें। & (अनुच्छेद - 51 क [ च, छ ]) \\ \hline 
57 & 4 & वैज्ञानिक हो दृष्टिकोण, ज्ञानार्जन का काम करें, & सावर्जनिक सम्पत्ति का, कभी नहीं नुकसान करें। & (अनुच्छेद - 51 क [ ज, झ ]) \\ \hline 
58 & 4 & खुद ऊंचाईयों को छू लें, लोगों को भी बढ़ायें, & राष्ट्र निरन्तर बढ़े आगे, सभी सफल हो जायें। & (अनुच्छेद - 51 क [ ञ ] ) \\ \hline 
59 & 4 & माता-पिता या संरक्षक, प्रथम काम ये करवायें, & चाहे गरीबी हो घर में, बच्चे को शिक्षा दिलवायें। & (अनुच्छेद - 51 क [ ट ] ) \\ \hline 
60 & 5 & भाग-5 & संघ  & (The Union) \\ \hline 
61 & 5 & A & कार्यपालिका  & (The Executive) \\ \hline 
62 & 5 & भारत सरकार में राष्ट्रपति, सर्वोच्चाधिकारी होगा, & जल थल वायु सेना का, वह शीर्षाधिकारी होगा। & (अनुच्छेद - 52, 53) \\ \hline 
63 & 5 & संसद विधानसभाओं द्वारा,राष्ट्रपति चयनित होगा, & पाँच साल का कार्यकाल,जिनका नियमित होगा। & (अनुच्छेद - 54, 56) \\ \hline 
64 & 5 & नागिरक हो भारत का, आयु पैंतीस से कम ना हो, & सांसद बनने लायक हों,कोई लाभ का पद ना हो। & (अनुच्छेद - 58) \\ \hline 
65 & 5 & मुख्य न्यायाधीश द्वारा, उनका शपथ ग्रहण होगा, & संविधान की रक्षा और अनुपालन का प्रण होगा। & (अनुच्छेद - 60) \\ \hline 
66 & 5 & राष्ट्रपति यदि संविधान की, अवहेलना करते हैं, & संसद सदस्य उनके ऊपर, दोषारोपण करते हैं। & (अनुच्छेद - 61) \\ \hline 
67 & 5 & संसद द्वारा दोषारोपण, महाभियोग कहलाता है, & दोष सिद्ध होने पर, उनको पद से हटाया जाता है। & (अनुच्छेद - 61) \\ \hline 
68 & 5 & राष्ट्रपति सर्वोच्च हैं, पर संसद है उनसे ऊपर, & राष्ट्रपति संसद न्यायालय, संविधान सबसे ऊपर। & (अनुच्छेद - 61 [भाव की अभिव्यक्ति]) \\ \hline 
69 & 5 & राष्ट्रपति यदि हों असमर्थ, अनुपस्थित या बीमार, & भारत के उपराष्ट्रपति, देखेंगे उनका कार्यभार। & (अनुच्छेद - 63, 65) \\ \hline 
70 & 5 & संसद के दो सदन करें, उपराष्ट्रपति का निर्वाचन, & संविधान के प्रति वे लेंगे, श्रद्धा, निष्ठा के वचन। & (अनुच्छेद - 66, 69) \\ \hline 
71 & 5 & राष्ट्रपति कर सकते हैं, किसी भी सजा को कम, & दे सकते हैं माफी भी, या फिर सजा का निलंबन। & (अनुच्छेद - 72) \\ \hline 
72 & 5 & राष्ट्रपति की मदद हेतु, एक मंत्री परिषद होगी, & समुचित कार्यवाही हेतु, जो अपनी सलाह देगी। & (अनुच्छेद - 74) \\ \hline 
73 & 5 & महामहिम को, सलाह का, पालन करना ही होगा, & इस सलाह की अनदेखी, संभव काम नहीं होगा। & (अनुच्छेद - 74) \\ \hline 
74 & 5 & मंत्री-मंडल की नियुक्ति, राष्ट्रपति द्वारा होगी, & लोकसभा के प्रति भी, मंडल की जिम्मेदारी होगी। & (अनुच्छेद - 75) \\ \hline 
75 & 5 & मंत्री-परिषद का मुखिया, प्रधानमंत्री होता है, & प्रजातंत्र में वह व्यक्ति, सबसे ताकतवर होता है। & (अनुच्छेद - 75) \\ \hline 
76 & 5 & एक महान्यायवादी होगा, भारत सरकार में नियुक्त, & सरकार को विधि सम्बन्धी, सलाह देगा उपयुक्त। & (अनुच्छेद - 76) \\ \hline 
77 & 5 & सरकार सब काम करेगी, राष्ट्रपति के नाम से, & उनको अवगत भी करायेगी, सरकार के काम से। & (अनुच्छेद - 77, 78) \\ \hline 
78 & 5 & B & संसद  & (Parliament) \\ \hline 
79 & 5 & संसद भारत देश की, सबसे बड़ी संस्था होगी, & नये कानून बनाने हेतु, उसकी ही क्षमता होगी। & (अनुच्छेद - 79) \\ \hline 
80 & 5 & तीन भागों से होता है, भारतीय संसद का गठन, & सर्वप्रथम हैं राष्ट्रपति, फिर संसद के दो सदन। & (अनुच्छेद - 79) \\ \hline 
81 & 5 & राज्य सभा के सदस्यों का, राज्य ही करें चयन, & कुछ विशेष लोगों का, किया जाता है मनोनयन। & (अनुच्छेद - 80) \\ \hline 
82 & 5 & प्रत्येक पाँच साल में होता, लोक सभा का चुनाव, & यही लोकतंत्र का जादू, जब लोग करे बदलाव। & (अनुच्छेद - 83) \\ \hline 
83 & 5 & राज्य सभा में नियुक्ति हेतु, उम्र चाहिए तीस, & लोकसभा में आने हेतु, पूरे करें पच्चीस। & (अनुच्छेद - 84) \\ \hline 
84 & 5 & राष्ट्रपति समय समय पर, करेंगे संसद में बैठक, & दोनों सदनों में अभिभाषण का, होगा उनका हक। & (अनुच्छेद - 86, 87) \\ \hline 
85 & 5 & लोकसभा शीघ्र चुनेगी, अपना एक अध्यक्ष, & लोकसभा के संचालन में, होगा वह अति दक्ष। & (अनुच्छेद - 93) \\ \hline 
86 & 5 & संसद के सब सदस्यों को, लेनी होगी एक शपथ, & संविधान का करेंगे पालन, अपनायेंगे उसका पथ। & (अनुच्छेद - 99) \\ \hline 
87 & 5 & सांसद नहीं बने रहेंगे, यदि हो विकृत मानसिकता, & हो जायें दिवालिया, या त्यागें देश की नागरिकता। & (अनुच्छेद - 102) \\ \hline 
88 & 5 & सांसद सदन में बोलें, या रखें स्वतंत्र विचार, & वाणी की स्वतंत्रता ही, है संसद का मुख्य आचार। & (अनुच्छेद - 105) \\ \hline 
89 & 5 & संसद बनायेगी कानून और पास करेगी बिल, & दोनों सदनों की सहमति से, काम नहीं मुश्किल। & (अनुच्छेद - 107) \\ \hline 
90 & 5 & जब कोई बिल पास होगा, सहमति देगें राष्ट्रपति, & यदि संविधान सम्मत है,तो जल्द ही देंगे स्वीकृति। & (अनुच्छेद - 111) \\ \hline 
91 & 5 & मंत्रालय सरकारी खर्चे का, पूरा बजट बनाता है, & वित्त मंत्री फिर इसको, संसद में पास कराता है। & (अनुच्छेद - 112) \\ \hline 
92 & 5 & संसद में होगी प्रयोग, हिन्दी या अंग्रेजी भाषा, & यदि कोई दोनों ना जाने, बोले अपनी मातृभाषा। & (अनुच्छेद - 120) \\ \hline 
93 & 5 & किसी जज के बर्ताव पर, संसद में ना होगी बहस, & संसदीय कार्यवाही पर, जज का नहीं चलेगा बस। & (अनुच्छेद - 121, 122) \\ \hline 
94 & 5 & संसद ना चल रही हो, परिस्थिति भी हो विशेष, & राष्ट्रपति कानून बनायें, जारी करके अध्यादेश। & (अनुच्छेद - 123) \\ \hline 
95 & 5 & C & न्यायपालिका  & (The Union Judiciary) \\ \hline 
96 & 5 & पूरे भारत में एक, सर्वोच्च न्यायालय होगा, & न्याय का मन्दिर होगा, जजों का कार्यालय होगा। & (अनुच्छेद - 124) \\ \hline 
97 & 5 & जज यदि बनना हो, तो चाहिए संविधान का ज्ञान, & वकील या जज के रूप में, पहले किया हो काम। & (अनुच्छेद - 124) \\ \hline 
98 & 5 & इन सभी जजों का चयन, राष्ट्रपति जी करते हैं, & इसके लिए अन्य जजों से, सलाह भी वह करते हैं। & (अनुच्छेद - 124) \\ \hline 
99 & 5 & पद ग्रहण से पूर्व, सब जज करते हैं ये वादा, & पक्षपात मैं नहीं करूंगा, रखूँ संविधान मर्यादा। & (अनुच्छेद - 124) \\ \hline 
100 & 5 & जज हो जायें असमर्थ, या तोड़े संविधान की हद, & संसद करेगी महाभियोग, छोड़ना होगा उनको पद। & (अनुच्छेद - 124) \\ \hline 
101 & 5 & सुप्रीम कोर्ट के पास, सबसे ज्यादा अधिकार, & मसला अपील का हो,या फिर राज्यों की तकरार। & (अनुच्छेद - 131, 132, 133, 134) \\ \hline 
102 & 5 & जरूरत पड़े तो सुप्रीम कोर्ट, नये कानून बनाता है, & जटिल संवैधानिक मसलों को भी,ये सुलझाता है। & (अनुच्छेद - 141, 143) \\ \hline 
103 & 5 & भारत में एक नियंत्रक महालेखापरीक्षक होगा, & केन्द्र-राज्य के खर्चों का, ब्यौरा उसे रखना होगा। & (अनुच्छेद - 148, 149) \\ \hline 
104 & 6 & भाग-6 & राज्य  & (The States) \\ \hline 
105 & 6 & A & कार्यपालिका  & (The Executive) \\ \hline 
106 & 6 & राज्यपाल हर राज्य में, सर्वोच्च अधिकारी होगा, & राष्ट्रपति करेंगे नियुक्ति,कार्यकाल पांच वर्ष होगा। & (अनुच्छेद - 153, 154, 155, 156) \\ \hline 
107 & 6 & राज्यपाल की नियुक्ति हेतु, भारत का नागिरक हो, & आयु पैंतीस की हो, अन्य लाभ का पद ना हो। & (अनुच्छेद - 157, 158) \\ \hline 
108 & 6 & राज्यपाल को मिलेंगे, भत्ते और विशेषािधकार, & साथ ही साथ होंगे वे, मुफ्त मकान के हकदार। & (अनुच्छेद - 158) \\ \hline 
109 & 6 & उच्च न्यायालय के जज से, राज्यपाल लेंगे शपथ, & जनता की करेंगे सेवा, अपनायेंगे संविधान पथ। & (अनुच्छेद - 159) \\ \hline 
110 & 6 & आपराधिक मामलों में, राज्यपाल रखते हैं शक्ति, & दोषी की कम करें सजा,या फिर दण्ड से दे मुक्ति। & (अनुच्छेद - 161) \\ \hline 
111 & 6 & राज्यपाल विवेकानुसार, करेंगे कार्यों का निर्वाह, & अन्य मामलों में वे लेंगे, मंत्री परिषद से सलाह। & (अनुच्छेद - 163) \\ \hline 
112 & 6 & राज्यपाल ही करेंगे, राज्य मुख्यमंत्री की नियुक्ति, & फिर उनकी सलाह पर, अन्य मंत्रियों की नियुक्ति। & (अनुच्छेद - 164) \\ \hline 
113 & 6 & राज्य के कुल मंत्रियों की, संख्या निर्धारित होगी, & एम एल ए की संख्या का,पंद्रह प्रतिशत ही होगी। & (अनुच्छेद - 164) \\ \hline 
114 & 6 & राज्य मंत्रियों की शपथ, संविधान के प्रति होगी, & सारी जनता की सेवा, पक्षपात रहित होगी। & (अनुच्छेद - 164) \\ \hline 
115 & 6 & मंत्री को गोपनीयता की, शपथ ग्रहण करनी होगी, & कुछ विषयों की जानकारी, उन्हें गुप्त रखनी होगी। & (अनुच्छेद - 164) \\ \hline 
116 & 6 & राज्यपाल ही नियुक्त करेंगे, राज्य का महाधिवक्ता, & कानूनी विषयों पर होगा,वो ही अधिकारिक वक्ता। & (अनुच्छेद - 165) \\ \hline 
117 & 6 & राज्य की कार्यवाही होगी, राज्यपाल के नाम से, & मुख्यमंत्री अवगत करायेंगे, उनको अपने काम से। & (अनुच्छेद - 166, 167) \\ \hline 
118 & 6 & B & विधान मण्डल  & (The State Legislature) \\ \hline 
119 & 6 & सब राज्यों में होगा, विधानसभा एक मुख्य सदन, & कुछ में अतिरिक्त होगा, विधानपरिषद का गठन। & (अनुच्छेद - 168) \\ \hline 
120 & 6 & विधानसभा में होगा, जनसंख्या का सही अनुपात, & ताकि हर समाज के लोग, पूरी रखें अपनी बात। & (अनुच्छेद - 170) \\ \hline 
121 & 6 & उम्र यदि पच्चीस की हो, मन में हो सेवा का भाव, & एम०एल०ए० बनने हेतु, लड़ सकते हो तुम चुनाव। & (अनुच्छेद - 173) \\ \hline 
122 & 6 & संविधान प्रति रखनी होगी, हर सदस्य को निष्ठा, & मेरा भारत रहे अखण्ड, ऊँची उसकी रहे प्रतिष्ठा। & (अनुच्छेद - 173) \\ \hline 
123 & 6 & विधानसभा अध्यक्ष, करेंगे सभा का संचालन, & हर दल रखे अपनी बात,हो नियमों का भी पालन। & (अनुच्छेद - 178) \\ \hline 
124 & 6 & सदन में हर सदस्य, अपनी बात कह सकता है, & बेखौफ, बेहिचक विचार, व्यक्त कर सकता है। & (अनुच्छेद - 194) \\ \hline 
125 & 6 & सदन, सदस्यों के वेतन को भी, तय कर सकता है, & जरूरत पड़े तो भत्तों में, बढ़ोत्तरी कर सकता है। & (अनुच्छेद - 195) \\ \hline 
126 & 6 & विधानसभा राज्य का बिल भी,पास कर सकती है, & फिर उस पर राज्यपाल से, स्वीकृति ले सकती है। & (अनुच्छेद - 196) \\ \hline 
127 & 6 & राज्यपाल सहमत होने पर, अपनी मुहर लगायेंगे, & बिल में कुछ गड़बड़ है, तो राष्ट्रपति को बतायेंगे। & (अनुच्छेद - 200) \\ \hline 
128 & 6 & विधानसभा बना सकती है, राज्य के लिए कानून, & जिसमें राज्य का हित हो, जनता को मिले सुकून। & (अनुच्छेद - 200) \\ \hline 
129 & 6 & राज्य बजट बतलायेगा,सरकार का वार्षिक खर्चा, & योजनाओं के मुद्दो पर, फिर होगी सदन में चर्चा। & (अनुच्छेद - 202, 203) \\ \hline 
130 & 6 & हिन्दी, अंग्रेजी, स्थानीय, होंगी सदन की भाषा, & कोई न जाने यदि इन्हें भी, बोलेगा वह मातृभाषा। & (अनुच्छेद - 210) \\ \hline 
131 & 6 & जज के आचरण पर, नहीं करेगा सदन बहस, & सदन की कार्यवाही पर भी, न चले कोर्ट का बस। & (अनुच्छेद - 211, 212) \\ \hline 
132 & 6 & विधानसभा का सत्र न हो, परिस्थिति हो विशेष, & राज्यपाल कानून बनायें, जारी करके अध्यादेश। & (अनुच्छेद - 213) \\ \hline 
133 & 6 & अध्यादेश को सदन द्वारा, करना होगा पास, & वर्ना होगा बेअसर वह, बीते यदि डेढ़ मास। & (अनुच्छेद - 21३) \\ \hline 
134 & 6 & C & राज्यों के उच्च न्यायालय & (The High Courts in the States) \\ \hline 
135 & 6 & हर राज्य में न्याय हेतु, एक उच्च न्यायालय होगा, & जिसमें जनता की खातिर, न्याय द्वार खुला होगा। & (अनुच्छेद - 214) \\ \hline 
136 & 6 & दस साल का न्यायिक अनुभव, या रहा हो वकील, & तब होगा उच्च न्यायालय का, जज बनने काबिल। & (अनुच्छेद - 217) \\ \hline 
137 & 6 & राष्ट्रपति द्वारा होगी, न्यायधीश की नियुक्ति, & बासठ वर्ष पूरा होने पर, होगी पद से मुक्ति। & (अनुच्छेद - 217) \\ \hline 
138 & 6 & न्यायधीश लेंगे शपथ, राज्यपाल के साथ, & सब लोगों को देंगे न्याय, बिना किए पक्षपात। & (अनुच्छेद - 219) \\ \hline 
139 & 6 & D & अधीनस्थ न्यायालय & (Subordinate Courts) \\ \hline 
140 & 6 & छोटे न्यायालयों पर होगी, हाईकोर्ट की निगरानी, & नियमपूर्वक कार्य करें सब, कोई न करे मनमानी। & (अनुच्छेद - 233) \\ \hline 
141 & 6 & जिला जजों की नियुक्ति, राज्यपाल द्वारा होगी, & इससे सम्बन्धित सलाह, हाईकोर्ट द्वारा होगी। & (अनुच्छेद - 233) \\ \hline 
142 & 6 & सात वर्ष वकालत वाला, जिला जज बनने योग्य, & अन्य न्यायिक पदों हेतु, है लोक सेवा आयोग। & (अनुच्छेद - 233) \\ \hline 
143 & 7 & भाग-7 & प्रथम अनूसूची के भाग ख में राज्य & (The States in Part B of FirstvSchedule) \\ \hline 
144 & 7 & A & निरसित  & (Repealed) \\ \hline 
145 & 8 & भाग-8 & संघ राज्यक्षेत्र & (The Union Territories) \\ \hline 
146 & 8 & संघ राज्यक्षेत्रों के लिए, प्रशासक होगें जिम्मेदार, & चाहे हो दमन, दीव या अंडमान और निकोबार। & (अनुच्छेद - 239) \\ \hline 
147 & 8 & राजधानी दिल्ली में है, उपराज्यपाल का शासन, & यहाँ विधानसभा भी है, जहाँ जनता करें निर्वाचन। & (अनुच्छेद - 239 क [ क ]) \\ \hline 
148 & 8 & भाग-9 & पंचायतें & (The Panchayats) \\ \hline 
149 & 8 & तेजी से हो सबका विकास, हो जनता का शासन, & इसके लिए करना होगा, पंचायतों का प्रशासन। & (अनुच्छेद - 243 ख) \\ \hline 
150 & 8 & ग्राम सभा हो, पंचायत हो और हो जिला परिषद, & अनुसूचित जाति, जनजाति हेतु आरक्षित हों पद। & (अनुच्छेद - 243 ग, घ) \\ \hline 
151 & 8 & महिलायें भी समाज की, हैं महत्वपूर्ण ईकाई, & उनके लिए आरक्षित हो, अध्यक्ष पद एक तिहाई। & (अनुच्छेद - 243 घ) \\ \hline 
152 & 8 & राज्य पंचायतो को देंगे, शक्ति और अधिकार, & विकास हो, न्याय मिले, हो जनता का उपकार। & (अनुच्छेद - 243 छ) \\ \hline 
153 & 8 & पंचायत कर सकती है, टैक्स वसूली का भी काम, & खर्चा-पूर्ति के लिए, राज्य भी देगा उन्हें अनुदान। & (अनुच्छेद - 243 ज) \\ \hline 
154 & 8 & प्रत्येक पाँच साल में होगा, पंचायत का निर्वाचन, & राज्य चुनाव आयोग करेगा, इन सबका संचालन। & (अनुच्छेद - 243 ट) \\ \hline 
155 & 9 & भाग-9 क & नगरपालिकायें & (The Municipalities) \\ \hline 
156 & 9 & शहरों में नगरपालिका, है पंचायत का दूसरा रूप, & वार्ड और उसकी समितियों जैसा होगा प्रारूप। & (अनुच्छेद - 243 थ, द, ध) \\ \hline 
157 & 9 & महिला, कमजोर वर्गों का, आरक्षण करना होगा, & योजना लागू करने को, टैक्स वसूल करना होगा। & (अनुच्छेद - 243 न, भ) \\ \hline 
158 & 9 & महानगर क्षेत्र में आबादी है, दस लाख से ज्यादा, & प्रमुख, मेयर कहलाता है,जिम्मेदारी भी है ज्यादा। & (अनुच्छेद - 243 त, थ, ब) \\ \hline 
159 & 9 & भाग-9 ख & सहकारी समितियां & (The Co-Operative Societies) \\ \hline 
160 & 9 & लोग बना सकते हैं, अपनी एक सहकारी समिति, & बोर्ड करेगा संचालन,बेहतर होगी आर्थिक स्थिति। & (अनुच्छेद - 243 यज, यञ) \\ \hline 
161 & 10 & भाग-10 & अनुसूचित और जनजाति क्षेत्र & (The Scheduled and Tribal Areas) \\ \hline 
162 & 10 & एस०टी० लोगों के विकास का, है विशेष प्रावधान, & अनुसूचित क्षेत्रों का भी, खूब रखना होगा ध्यान। & (अनुच्छेद - 244) \\ \hline 
163 & 10 & जनजाति के लिए, एक सलाहकार परिषद होगी, & जिससे उनके प्रशासन की, नीति निर्धारित होगी। & (अनुच्छेद - 244, पाँचवी अनुसूची) \\ \hline 
164 & 10 & उनके कल्याण कार्यों पर, राज्यपाल रखें निगरानी & भूमि उनकी रहे सुरक्षित, साहूकार न करें मनमानी। & (अनुच्छेद - 244, पाँचवी अनुसूची) \\ \hline 
165 & 10 & राज्यपाल उनके क्षेत्रों में, परिवर्तन कर सकते हैं, & शान्ति सुरक्षा कानूनों में, बदली भी कर सकते हैं। & (अनुच्छेद - 244, पाँचवी अनुसूची) \\ \hline 
166 & 10 & असम राज्य में होंगे, स्वशासी जिला और प्रदेश, & लागू नहीं होंगे इन पर, राज्य सरकार के निर्देश। & (अनुच्छेद - 244, छठी अनुसूची) \\ \hline 
167 & 10 & जंगल और जनजाति में है, बहुत गहरा संबंध, & संतुलन कैसे रखा जाये, इसका है सही प्रबंध। & (अनुच्छेद - 244, छठी अनुसूची) \\ \hline 
168 & 10 & वनभूमि का, उत्पादों का, जनजाति करे प्रयोग, & वन भी रहे सुरक्षित, ना हो जीविका में अवरोध। & (अनुच्छेद - 244, छठी अनुसूची) \\ \hline 
169 & 10 & राज्यपाल ऐसे क्षेत्रों में, निश्चित करें सुशासन, & शिक्षा, स्वास्थ्य, संचार के दिलवाये सब साधन। & (अनुच्छेद - 244, छठी अनुसूची) \\ \hline 
170 & 11 & भाग-11 & संघ और राज्यों के बीच में सम्बन्ध & (Relations between the Union and the States) \\ \hline 
171 & 11 & संसद पूरे भारत में, कानून बना सकती है, & विधानसभा केवल राज्य में, ऐसा कर सकती है। & (अनुच्छेद - 245) \\ \hline 
172 & 11 & कानून बनाने की शक्ति, केन्द्र-राज्य की है पृथक, & विषय उनके निर्धारित हैं, ताकि ना हो कोई शक। & (अनुच्छेद - 246) \\ \hline 
173 & 11 & संघ सूची में लिखे हैं, संसद के अधिकार, & जैसे रक्षा, रेल, विदेश, सेना, दूर संचार। & (अनुच्छेद - 246, सातवीं अनुसूची - संघ सूची) \\ \hline 
174 & 11 & राज्य सूची में वर्णित है, विधानसभा के अधिकार, & जैसे पुलिस स्वास्थ्य कृषि, वन विधुत कारागार। & (अनुच्छेद - 246, सातवीं अनुसूची - राज्य सूची) \\ \hline 
175 & 11 & कुछ विषय ऐसे हैं, जिन पर दोनों का अधिकार, & इनमें से हैं शिक्षा, सजा, न्याय, वन, व्यापार। & (अनुच्छेद - 246, सातवीं अनुसूची - समवर्ती सूची) \\ \hline 
176 & 11 & फिर भी संसद बड़ी है, श्रेष्ठ हैं उसके अधिकार, & राष्ट्रहित में कानून बनाये, करे अन्तराष्ट्रीय करार। & (अनुच्छेद - 247, 248, 249, 250, 251, 252, 253, 254) \\ \hline 
177 & 11 & कानूनों का अनुपालन निश्चित करे राज्य सरकार, & वर्ना निर्देशित भी कर सकती है, भारत सरकार। & (अनुच्छेद - 256) \\ \hline 
178 & 11 & केंद्र, राज्य सरकारों को, दे सकता है आदेश, & देशहित के कामों में, यदि मुश्किल आती है पेश। & (अनुच्छेद - 257) \\ \hline 
179 & 11 & दो राज्यों में विवाद हो, या नदी जल का बँटवारा, & केन्द्र कानून बनाकर, कर सकता है निपटारा। & (अनुच्छेद - 262) \\ \hline 
180 & 11 & राज्यों के झगड़ों को, सब मिलजुल के सुलझायें, & जाँच एवं सुझाव हेतु, परिषद एक बनायें। & (अनुच्छेद - 263) \\ \hline 
181 & 12 & भाग-12 & वित्त, सम्पत्ति, संविदाएँ और वाद & (Finance, Property, Contracts and Suits) \\ \hline 
182 & 12 & टैक्स द्वारा जो पैसा आये, संचित धन कहलाये, & भाँति-भाँति के खर्चों में, यह धन काम में आये। & (अनुच्छेद - 266) \\ \hline 
183 & 12 & कुछ खर्चे करने पड़ते हैं, एकदम और कई बार, & जिसकी खातिर कंटीजेंसी फंड,रखती हैं सरकार। & (अनुच्छेद - 267) \\ \hline 
184 & 12 & केंद्र और राज्य सरकारें, दोनों लें जनता से कर, & फिर नियमानुसार आपस में, बाँटें इसे बराबर। & (अनुच्छेद - 268, 269, 270) \\ \hline 
185 & 12 & केन्द्र को, राज्य सरकारों को, देना होगा अनुदान, & जनजाति क्षेत्रों के विकास का, रखना होगा ध्यान। & (अनुच्छेद - 275) \\ \hline 
186 & 12 & वित्त आयोग सुझाव देता है, धन के वितरण पर, & ताकि राज्यों को मिले, उन्नित के बराबर अवसर। & (अनुच्छेद - 280, 281) \\ \hline 
187 & 12 & केंद्र की सम्पत्ति पर, राज्य नहीं लगायें कर, & केंद्र भी इस सम्बन्ध में, करे राज्य का आदर। & (अनुच्छेद - 285, 289) \\ \hline 
188 & 12 & केंद्र, राज्य कर सकते हैं, व्यापार और कारोबार, & समुद्री खनिजों पर होगा,केवल केंद्र का अधिकार। & (अनुच्छेद - 297, 298) \\ \hline 
189 & 12 & राष्ट्रपति की ओर से, साइन होंगे सब करार, & पर कोई चूक हुई तो, नहीं होंगे वे जिम्मेदार। & (अनुच्छेद - 299) \\ \hline 
190 & 12 & कोई व्यक्ति कर सकता है, सम्पत्ति को संचित, & कानून बिना उसे, नहीं कर सकते हैं वंचित। & (अनुच्छेद - 300 क) \\ \hline 
191 & 13 & भाग-13 & भारत के राज्य क्षेत्र के अन्दर व्यापार, वाणिज्य और समागम & (Trade, Commerce and Intercourse in India) \\ \hline 
192 & 13 & कोई कहीं भी कर सकता है, वाणिज्य या व्यापार, & जिसे जनहित में नियमित, कर सकती है सरकार। & (अनुच्छेद - 301, 302) \\ \hline 
193 & 13 & केन्द्र फंड के वितरण में, न करे किसी से पक्षपात, & साधन विहीन राज्यों का,फिर भी देना होगा साथ। & (अनुच्छेद - 303) \\ \hline 
194 & 14 & भाग-14 & संघ और राज्यों के अधीन सेवाएँ & (Services under the Union and the States) \\ \hline 
195 & 14 & सरकारी कामकाज हेतु, नियुक्त होते हैं अधिकारी, & कार्य सुचारू रूप से हो, लेते हैं जिम्मेदारी। & (अनुच्छेद - 309) \\ \hline 
196 & 14 & कोई अधिकारी कर्मठ है, काम है उसका अच्छा, & संविधान उसको देता है, काफी अधिक सुरक्षा। & (अनुच्छेद - 310, 311) \\ \hline 
197 & 14 & नियुक्ति कर्ता ही किसी को, पद से हटा सकता है, & विधिवत जाँच कराके, सजा सुना सकता है। & (अनुच्छेद - 311) \\ \hline 
198 & 14 & किसी कार्यवाही से पहले, व्यक्ति को सुनना होगा, & उसके दोष बताने होंगे, फिर निर्णय करना होगा। & (अनुच्छेद - 311) \\ \hline 
199 & 14 & यदि गंभीर अपराध में, दोषी सिद्ध हुआ अधिकारी, & पद से हटाया जा सकता है, उसे बिना इन्क्वारी। & (अनुच्छेद - 311, 2 क) \\ \hline 
200 & 14 & कभी-कभी सुरक्षा हित में, जाँच नहीं हो सकती है, & बिना जाँच, अधिकारी की सेवामुक्ति हो सकती है। & (अनुच्छेद - 311, 2 ख, ग) \\ \hline 
201 & 14 & केंद्र में भर्ती हेतु है, संघ लोक सेवा आयोग, & आवेदन कर सकते हैं, सेवा के इच्छुक सब लोग। & (अनुच्छेद - 315) \\ \hline 
202 & 14 & आयोग करेगा संचालित, नियुक्ति हेतु परीक्षा, & उच्च सेवाओं में चाहिए, ग्रेजुएट तक की शिक्षा। & (अनुच्छेद - 315, 320) \\ \hline 
203 & 14 & राज्यों में भर्ती हेतु है, राज्य लोक सेवा आयोग, & छोटे राज्य चाहें तो, बनेगा एक संयुक्त आयोग। & (अनुच्छेद - 315) \\ \hline 
204 & 14 & आयोग में होंगे सदस्य, होगा एक अध्यक्ष, & रहें हो सरकारी पद पर, छवि हो उनकी निष्पक्ष। & (अनुच्छेद - 316) \\ \hline 
205 & 14 & आयोग, भर्ती सम्बन्धित, अपनी सलाह भी देगा, & राष्ट्रपति को हर साल, सब कार्यों का ब्यौरा देगा। & (अनुच्छेद - 323) \\ \hline 
206 & 14 & भाग-14क & अधिकरण  & (Tribunals) \\ \hline 
207 & 14 & केंद्र, राज्य स्तर पर होंगे, प्रशासनिक अधिकरण, & भर्ती, सेवा के झगड़ों का, किया करेंगे निराकरण। & (अनुच्छेद - 323 क) \\ \hline 
208 & 14 & कुछ अधिकरण करेंगे, अन्य विवादों का निपटारा, & टैक्स, श्रम विवाद, या खाद्य वस्तुओं का बँटवारा। & (अनुच्छेद - 323 ख) \\ \hline 
209 & 15 & भाग-15 & निर्वाचन  & (Elections) \\ \hline 
210 & 15 & प्रजातन्त्र का मुख्य तन्त्र है, निर्वाचन आयोग, & इसकी मदद से सरकारों को, चुनते हैं हम लोग। & (अनुच्छेद - 324) \\ \hline 
211 & 15 & मुख्य चुनाव अायुक्त करायें, चुनाव का संचालन, & आचार संहिता का करवायें, सख्ती से अनुपालन। & (अनुच्छेद - 324) \\ \hline 
212 & 15 & राष्ट्रपति, उपराष्ट्रपति, संसद या विधानमंडल, & सबके चुनाव पर आयोग, निगरानी रखे प्रतिपल। & (अनुच्छेद - 324) \\ \hline 
213 & 15 & चुनाव आयुक्तों को देता है, संविधान सुरक्षा, & स्वतन्त्र चुनावों से ही होती, प्रजातन्त्र की रक्षा। & (अनुच्छेद - 324) \\ \hline 
214 & 15 & लोकसभा, विधान सभा का, जब भी हो निर्वाचन, & आयोग को सब सहयोग, देगा राज्य प्रशासन। & (अनुच्छेद - 324) \\ \hline 
215 & 15 & वोटर लिस्ट में शामलि होगा, हर वयस्क का नाम, & इस सब को निष्पक्ष कराना, है आयोग का काम। & (अनुच्छेद - 325) \\ \hline 
216 & 15 & अठारह वर्ष का हर व्यक्ति, रखता है मताधिकार, & अपनी इच्छा के अनुसार, चुन सकता है सरकार। & (अनुच्छेद - 326) \\ \hline 
217 & 15 & एम०पी० और एम०एल०ए० बने, हो चुनाव निष्पक्ष, & लोकतंत्र कायम रहे, यही आयोग का लक्ष्य। & (अनुच्छेद - 324 [भाव की अभिव्यक्ति]) \\ \hline 
218 & 16 & भाग-16 & कुछ वर्गों के सम्बन्ध में विशेष उपबन्ध & (Special Provisions relating to Certain Classes) \\ \hline 
219 & 16 & अनुसूचित जाति, जनजाति का है विशेष प्रबंध, & लोकसभा में है आरक्षण, है अनुपातिक संबंध। & (अनुच्छेद - 330) \\ \hline 
220 & 16 & ऐंग्लो इंिडयन लोग यदि, संख्या में कम लगते हैं, & राष्ट्रपति लोकसभा में, दो सदस्य रख सकते हैं। & (अनुच्छेद - 331) \\ \hline 
221 & 16 & एससी, एसटी लोगों की, िजतनी जनसंख्या होगी, & उतनी ही, विधानसभा में, सीटों की संख्या होगी। & (अनुच्छेद - 332) \\ \hline 
222 & 16 & सरकारी सेवाओं में,एससी, एसटी का हो स्थान, & थोड़ी छूट मिले अंकों में, ताकि भर्ती हो आसान। & (अनुच्छेद - 335) \\ \hline 
223 & 16 & राष्ट्रीय अनुसूचितजाति आयोग करे हकों की रक्षा & उनसे संबंधित समस्याओं की, करता है ये परीक्षा। & (अनुच्छेद - 338) \\ \hline 
224 & 16 & एक अध्यक्ष, उपाध्यक्ष और अन्य सदस्य होंगे, & संविधान के प्रावधान, उन्हें लागू करने होंगे। & (अनुच्छेद - 338) \\ \hline 
225 & 16 & एस० सी० लोगों के हितों में, न हो कोई कोताही, & साबित हो आरोप यदि, आयोग करे कार्यवाही। & (अनुच्छेद - 338) \\ \hline 
226 & 16 & अन्य कल्याणकारी, उपाय खोजकर लायेगा, & राष्ट्रपति को आयोग, अपने सुझाव बतलायेगा। & (अनुच्छेद - 338) \\ \hline 
227 & 16 & आयोग का दर्जा, सिविल न्यायालय का होगा, & सबकी पेशी कराने का हक, उसे प्राप्त होगा। & (अनुच्छेद - 338) \\ \hline 
228 & 16 & राष्ट्रपति आयोग की रिपोर्ट, संसद में रखवायेंगे, & सिफारिशों पर कारर्वाई से, अवगत भी करवायेंगे। & (अनुच्छेद - 338) \\ \hline 
229 & 16 & आयोग चाहे, तो संबंधित कागज, पेश करने होंगे, & शपथपत्र देकर सारे सबूत, प्रकट करने होंगे। & (अनुच्छेद - 338) \\ \hline 
230 & 16 & एस० सी० संबंधित, कोई निर्णय लेती है सरकार, & आयोग से लेकर सलाह, उसे करना होगा विचार। & (अनुच्छेद - 338) \\ \hline 
231 & 16 & एक राष्ट्रीय अनुसूचित जनजाति आयोग भी होगा & एस सी आयोग जैसा स्तर, इसको भी प्राप्त होगा। & (अनुच्छेद - 338 क) \\ \hline 
232 & 16 & वैसी ही शक्ति होगी, वही कार्य वह करेगा, & जनजाति के हितों की रक्षा, हर हालत में करेगा। & (अनुच्छेद - 338 क) \\ \hline 
233 & 16 & एस०टी० की उन्नित हेतु, आयोग की होगी नियुक्ति & राज्य यदि करे ढिलाई, केंद्र करेगा सख्ती। & (अनुच्छेद - 339) \\ \hline 
234 & 16 & सामाजिक, शैक्षिक दृष्टि से, पिछड़े हैं जो लोग, & राष्ट्रपति उनके लिए, गठित करें आयोग। & (अनुच्छेद - 340) \\ \hline 
235 & 16 & आयोग अध्ययन करेगा, कैसे हो उनका सुधार, & कैसे केंद्र-राज्य बन सकते, उन लोगों के मददगार। & (अनुच्छेद - 340) \\ \hline 
236 & 16 & आयोग राष्ट्रपति को देगा, सिफारिशें ब्यौरेवार, & संसद को फिर करना होगा, भलाई हेतु विचार। & (अनुच्छेद - 340) \\ \hline 
237 & 17 & भाग-17 & राजभाषा  & (Official Language) \\ \hline 
238 & 17 & संघ की भाषा हिंदी, लिपि देवनागरी होगी, & शासकीय प्रयोजन हेतु, अंग्रेजी की जरूरत होगी। & (अनुच्छेद - 343) \\ \hline 
239 & 17 & राष्ट्रपति भाषाविदों का, एक आयोग करें गठित, & हिंदी के प्रचार हेतु, प्रयास करे जो संगठित। & (अनुच्छेद - 344) \\ \hline 
240 & 17 & हिंदी भाषा हो समृद्ध, बढ़े ज्ञान और विज्ञान, & अहिंदी भाषी लोगों का, आयोग रखे ध्यान। & (अनुच्छेद - 344) \\ \hline 
241 & 17 & विधानसभा स्थानीय भाषा को, कर ले अंगीकार, & पर सरकारी कामों में, रहे अंग्रेजी बरकरार। & (अनुच्छेद - 345) \\ \hline 
242 & 17 & दो राज्यों को आपस में, यदि करना है पत्राचार, & दोनों मिलकर हिंदी को, कर सकते हैं स्वीकार। & (अनुच्छेद - 346) \\ \hline 
243 & 17 & उच्च, उच्चतम न्यायालय में, अंग्रेजी प्रयुक्त होगी, & आदेशों,अधिनियमों की प्रति, अंग्रेजी में ही होगी। & (अनुच्छेद - 348) \\ \hline 
244 & 17 & भाषाई अल्पसंख्यकों का, सरकार रखे ध्यान, & मातृभाषा में ही मिले, सब विषयों का ज्ञान। & (अनुच्छेद - 350 क, ख) \\ \hline 
245 & 17 & सरकार हिंदी को करे समृद्ध, खूब करे प्रचार, & अन्य भाषाओं के शब्दों से, करे इसका विस्तार। & (अनुच्छेद - 351) \\ \hline 
246 & 18 & भाग-18 & आपात उपबन्ध & (Emergency Provisions) \\ \hline 
247 & 18 & युद्ध का हो संकट, विद्रोहियों का बिछा हो जाल, & राष्ट्रपति सन्तुष्टि करके,घोषित करें आपातकाल। & (अनुच्छेद - 352) \\ \hline 
248 & 18 & आपातकाल के लिए चाहिए, संसद की मंजूरी, & एक माह के भीतर ही, यह सहमति अति जरूरी। & (अनुच्छेद - 352) \\ \hline 
249 & 18 & इस दौरान मिल जाते हैं,केंद्र को ज्यादा अधिकार, & केंद्र सरकार चला सकती है,राज्य की भी सरकार। & (अनुच्छेद - 353) \\ \hline 
250 & 18 & केंद्र, राज्य की करेगा, हर हालत में सुरक्षा, & चाहे वाह्य आक्रमण हो या सशस्त्र विद्रोह से रक्षा। & (अनुच्छेद - 355) \\ \hline 
251 & 18 & राज्य यदि नहीं चल रहा, संविधान के अनुसार, & राज्यपाल राष्ट्रपति को देंगे, रिपोर्ट सिलसिलेवार। & (अनुच्छेद - 356) \\ \hline 
252 & 18 & फिर उस राज्य में लागू होगा, राष्ट्रपति शासन, & भारत सरकार हाथ में लेगी, वहाँ का प्रशासन। & (अनुच्छेद - 356) \\ \hline 
253 & 18 & उदघोषणा का संसद द्वारा, करना होगा अनुमोदन, & दो महीनों में पास करेंगे, संसद के दोनों सदन। & (अनुच्छेद - 356) \\ \hline 
254 & 18 & आपातकाल में छिन जायेंगे, सारे मूल अधिकार, & बचेगी शारीरिक स्वतंत्रता व जीने का अधिकार। & (अनुच्छेद - 358, 359) \\ \hline 
255 & 18 & भारत में यदि पैदा हो, आर्थिक संकट काल, & राष्ट्रपति घोषित करते हैं, वित्तीय आपातकाल। & (अनुच्छेद - 352) \\ \hline 
256 & 18 & इस दौरान राज्यों के खर्चे, किये जायेंगे कम, & घट जायेंगे कमर्चािरयों और जजों के वेतन। & (अनुच्छेद - 360) \\ \hline 
257 & 19 & भाग-19 & प्रकीर्ण  & (Miscellaneous) \\ \hline 
258 & 19 & राष्ट्रपति व राज्यपाल को, प्राप्त हैं विशेषाधिकार, & ना उन पर कोई केस चलेगा, न होंगे वे गिरफ्तार। & (अनुच्छेद - 361) \\ \hline 
259 & 19 & यदि कोई छपवाता है, संसद की कार्यवाही, & न्यायालय नही करेगा, कोई दाण्डिक कार्यवाही। & (अनुच्छेद - 361 क) \\ \hline 
260 & 20 & भाग-20 & संविधान का संशोधन  & (Amendment of the Constitution) \\ \hline 
261 & 20 & संसद चाहे, कर सकती है, संविधान में संशोधन, & दो तिहाई बहुमत से, यदि पास करें दोनों सदन। & (अनुच्छेद - 368) \\ \hline 
262 & 20 & फिर ये विधेयक, राष्ट्रपति को भेजा जायेगा, & उनकी स्वीकृति मिलने पर, संशोधन कहलायेगा। & (अनुच्छेद - 368) \\ \hline 
263 & 21 & भाग-21 & अस्थायी, संक्रमणकालीन और विशेष उपबन्ध & (Temporary, Transitional and Special Provisions) \\ \hline 
264 & 21 & जम्मू और कश्मीर राज्य का, है विशेष अधिकार, & संसद का कानून चलेगा, जब राज्य करे स्वीकार। & (अनुच्छेद - 370) \\ \hline 
265 & 21 & कई बहुत से राज्यों में भी, हैं विशेष उपबन्ध, & उनकी उन्नित और विकास हो, कुछ ऐसे प्रबंध। & (अनुच्छेद - 371, 371 क, ख, ग, घ, ङ, च, छ, ज, झ, ञ) \\ \hline 
266 & 22 & भाग-22 & संक्षिप्त नाम, प्रारम्भ, हिन्दी में प्राधिकृत पाठ & (Short Title, Commencement, Authoritative Text in Hindi) \\ \hline 
267 & 22 & हमारा ये संविधान,भारत का संविधान कहलायेगा, & छब्बीस जनवरी 1950 से, पूरा लागू हो जायेगा। & (अनुच्छेद - 393, 394) \\ \hline 
268 & 22 & राष्ट्रपति जी संविधान को, हिन्दी में छपवायेंगे। & हिन्दी, अंग्रेजी संस्करण ही, अधिकृत कहलायेंगे। & (अनुच्छेद - 394 क) \\ \hline 
 \end{longtable}
